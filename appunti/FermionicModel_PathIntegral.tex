\documentclass[a4paper,11pt]{amsart}


\usepackage{amsmath, amssymb, amsthm}
\usepackage{mathtools}
\usepackage{vmargin}

\usepackage[all]{xy}
\usepackage{graphicx}

\usepackage[utf8]{inputenc}
\usepackage[T1]{fontenc}   
\usepackage[english]{babel}
\usepackage{fullpage}
\usepackage{hyperref}

\usepackage{enumitem}
\usepackage{xfrac}
\usepackage{slashed}
\usepackage{cancel}

\usepackage{afterpage}

\newcommand\blankpage{%
    \null
    \thispagestyle{empty}%
    \addtocounter{page}{-1}%
    \newpage}

%-------------------------------------------------------------------------------

\allowdisplaybreaks[4]

%-------------------------------------------------------------------------------

\newcommand{\ssection}[2]{\section{ \texorpdfstring{\textbf{#1}}{#2} }}

\newcommand{\HRule}{\center{\rule{0.9\linewidth}{0.5mm}}}

% add these definistions to the definitions' file!!!!!!! --------------------------------------

\DeclareMathOperator{\sign}{sign}
\DeclareMathOperator{\Tr}{Tr}
\DeclareMathOperator{\tr}{tr}

\newcommand{\mean}[1]{\ensuremath{\langle #1 \rangle}}
\newcommand{\kett}[1]{\ensuremath{\left | #1 \right \rangle}}
\newcommand{\brat}[1]{\ensuremath{\left \langle  #1 \right | }}
\newcommand{\ket}[1]{\left | #1 \right \rangle}
\newcommand{\bra}[1]{\left \langle  #1 \right |}

\newcommand{\nl}{\vskip 0.3cm}
\newcommand{\np}{\vskip 1.3cm}

\newcommand{\obar}[1]{\overline{#1}}
\newcommand{\ubar}[1]{\underline{#1}}
\newcommand{\psibar}{\bar{\psi}}
\newcommand{\sigmatilde}{\tilde{\sigma}}

\newcommand{\der}[2]{\frac{\partial #1}{\partial #2}}
\newcommand{\demu}[1]{\partial#1{\mu}}
\newcommand{\denu}[1]{\partial#1{\nu}}
\newcommand{\pder}[3]{\dfrac{\partial^{#1} #2}{\partial^{#1} #3}}
\newcommand{\de}[1]{\frac{d^2#1}{(2\pi)^2}}
 
% ---------------------------------------------------------------------------------------------


% ---- info ---------
\title{Notes: Fermionic model of interest}
\author{Federico Tonielli}

% ---- document -----
\begin{document}

 \maketitle
 
 \tableofcontents
 \blankpage
 
 \section{The second-quantized model}
 Here we present the general model under study in a second quantized formalism.  Moreover, the full equation of motion of the two-point correlation matrix at equal times is derived.
  \subsection{Quantum Master Equation in Majorana form}
  We consider an open quantum system of fermions on a D-dimensional lattice; we assume its dynamics to be described by a Markovian Quantum Master Equation (QME hereby), which in the Lindblad form reads
  \[  \]
  Here 
  \subsection{Equation of motion for the Covariance Matrix from QME}
 \section{Path Integral reformulation of the QME}
  \subsection{Keldysh action for Majorana fermions}
  \subsection{Re-derivating the equation for the stationary Covariance Matrix}

  \appendix
  \section*{Appendix}
    \renewcommand{\thesection}{A}
    \subsection{``Majorana doubling'' trick}
   
   
\bibliography{library}{}
\bibliographystyle{amsplain}  
\end{document}
