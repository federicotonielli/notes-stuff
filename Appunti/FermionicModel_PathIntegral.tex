\documentclass[a4paper,11pt]{article}

\usepackage{amsmath, amssymb, amsthm}
\usepackage{mathtools}
\usepackage{vmargin}

\usepackage[all]{xy}
\usepackage{graphicx}
\usepackage{framed}

\usepackage[utf8]{inputenc}
\usepackage[T1]{fontenc}   
\usepackage[english]{babel}
% \usepackage{fullpage}
\usepackage{hyperref}

\usepackage{enumitem}
\usepackage{xfrac}
\usepackage{slashed}
\usepackage{color}
\usepackage{cancel}

\usepackage{afterpage}

\newcommand\blankpage{%
    \null
    \thispagestyle{empty}%
    \addtocounter{page}{-1}%
    \newpage}

%-------------------------------------------------------------------------------

\allowdisplaybreaks[4]

%-------------------------------------------------------------------------------

\newcommand{\ssection}[2]{\section{ \texorpdfstring{\textbf{#1}}{#2} }}

\newcommand{\HRule}{\center{\rule{0.9\linewidth}{0.5mm}}}

% add these definistions to the definitions' file!!!!!!! --------------------------------------

\DeclareMathOperator{\sign}{sign}
\DeclareMathOperator{\Tr}{Tr}
\DeclareMathOperator{\tr}{tr}
\DeclareMathOperator{\Det}{Det}

\theoremstyle{remark}
\newtheorem{remark}{Remark}

\newcommand{\mean}[1]{\ensuremath{\langle #1 \rangle}}
\newcommand{\kett}[1]{\ensuremath{\left | #1 \right \rangle}}
\newcommand{\brat}[1]{\ensuremath{\left \langle  #1 \right | }}
\newcommand{\ket}[1]{\left | #1 \right \rangle}
\newcommand{\bra}[1]{\left \langle  #1 \right |}
\newcommand{\ro}{\rho}
\newcommand{\ra}{\rightarrow}

\newcommand{\nl}{\vskip 0.3cm}
\newcommand{\ns}{\vskip 0.8cm}
\newcommand{\np}{\vskip 1.3cm}

\newcommand{\obar}[1]{\overline{#1}}
\newcommand{\ubar}[1]{\underline{#1}}
\newcommand{\psibar}{\bar{\psi}}
\newcommand{\sigmatilde}{\tilde{\sigma}}

\newcommand{\der}[2]{\frac{\partial #1}{\partial #2}}
\newcommand{\demu}[1]{\partial#1{\mu}}
\newcommand{\denu}[1]{\partial#1{\nu}}
\newcommand{\pder}[3]{\dfrac{\partial^{#1} #2}{\partial^{#1} #3}}
\newcommand{\de}[1]{\frac{d^2#1}{(2\pi)^2}}

\newcommand*{\mathcolor}{}  %allows to change the colour inside math mode without destroying the correct spacing like with \textcolor
\def\mathcolor#1#{\mathcoloraux{#1}}
\newcommand*{\mathcoloraux}[3]{%
  \protect\leavevmode
  \begingroup
    \color#1{#2}#3%
  \endgroup
}
 
% ---------------------------------------------------------------------------------------------


% ---- info ---------
\title{Notes: Free Fermions with quadratic dissipation}
\author{Federico Tonielli}

% ---- document -----
\begin{document}

 \maketitle
 
 \tableofcontents
 \blankpage
 
 \section{The second-quantized model}
 Here we present the general model under study in a second quantized formalism.  Moreover, the full equation of motion of the two-point correlation function at equal times is derived.
  \np
  \subsection{Quantum Master Equation in Majorana form}
   We consider an open quantum system of Dirac fermions on a lattice, which we will consider finite for simplicity; the creation and annihilation operators for these fermions are $\left \{ a_i^{ },a_i^+, i\in \{1\dots N\} \right \} $\footnote{ The usual hat to distinguish between operators and numbers/fields will be omitted when the distinction is clear from the context; numbers (matrix elements) will be referred to using capital letters.} and they obey the usual anticommutation relations $\left\{a_i^{ },a_j^{ }\right\} = \left\{a_i^{+},a_j^+\right \}=0$ and $\left\{a_i^{ },a_j^+\right \} = \delta_{ij}\,id $.\\  We assume its dynamics to be described by a Markovian Quantum Master Equation (herein after referred as QME), which in the Lindblad form reads
   \[\dot{\ro}=\mathcal{L}[\ro]=-i\left[H,\ro\right]+\sum_{\alpha}\left[L_{\alpha}^{ }\ro L_{\alpha}^+ - \frac{1}{2}L_{\alpha}^+L_{\alpha}^{ }\ro - \frac{1}{2}\ro L_{\alpha}^+L_{\alpha}^{ }\right]  \]
   Furthermore, we assume the Hamiltonian to be quadratic and the Lindblad operators to be linear in the fermionic fields, without any other constraint: 
   \begin{subequations}        %sub-numbering of the equations which follow
%   \renewcommand{\theequation}{\theparentequation.\arabic{equation}}         %sub-numbering is 1.1, 1.2 instead of standard 1a, 1b
   \label{eq:quadr_dyn}         %ref the global set of equations as (1) with \eqref{eq:global_label}
   \begin{align}
    H & = a_j^+M_{jk}a_k^{ } \equiv a^+Ma \label{eq:quadr_ham}\\         %ref the first eq as (1a) with \eqref{eq:local_1st_label}
    L_{\alpha} & = P^{ }_{\alpha j}a_j^{ } + Q^{ }_{\alpha j}a_j^+ \Rightarrow L \equiv P a + Q a^+ \label{eq:quadr_lindb}
   \end{align}
   \end{subequations}
   A compact notation has been introduced in order to simplify the form of next calculations: 
   \begin{itemize}
   \item indices are omitted and the order of symbols in the expressions also indicates the order of index contraction (e.g. $Mc\rightarrow M_{ij}a_j$ and $cM \rightarrow a_j M_{ji}$);
   \item operators and matrices of numbers appear on the same footing in the formulae, according to the already mentioned rules of notation; there is no need to distinguish between index-carrying and index-free operators/numbers as it will be always clear from the context.
   \end{itemize}
   \ns
   Following \cite{Eisert2010}, it is convenient to describe the system using Majorana fermionic operators:
   \begin{subequations}
   \label{eq:def_majorana}
   \begin{align}
   &\begin{aligned}
    &c^{ }_{j, 1} = a_j^{ }+a_j^{+} \\
    &c^{ }_{j, 2} = i(a_j^{ }-a_j^{+})
   \end{aligned}
   &\longleftrightarrow&
   & &\begin{aligned}
    &a_j^{ }=\frac{1}{2}\left(c^{ }_{j, 1}-ic^{ }_{j, 2}\right)\\
    &a_j^+=\frac{1}{2}\left(c^{ }_{j, 1}+ic^{ }_{j, 2}\right)
   \end{aligned}\\[0.3 cm]
   &\left\{c^{ }_{i,r},c^{ }_{j,s}\right\} = 2\delta_{ij}\delta_{rs}
   &\longleftrightarrow&
   & &\begin{aligned}
    &\left\{a^{ }_{i},a^{+}_{j}\right\} = \delta_{ij} \\
    &\big\{a^{ }_{i},a^{ }_{j}\big\} = 0 \\
    &\left\{a^{+}_{i},a^{+}_{j}\right\} = 0 
    \end{aligned}
   \end{align}
   \end{subequations}\np
   To what concerns us, they have these important properties:
   \begin{itemize}
   \item they are self-adjoint;
   \item they square to $id$, so they have no vacuum nor any Grassmann coherent state as an eigenstate: if they had, these would be vectors $\ket{v}$ such that $c^2\ket{v}=0$ (for the vacuum or Grassmann property), but $c^2\ket{v}=id\ket{v}=\ket{v}\neq 0$;
   \item their algebra can be straightforwardly mapped into a 2N-dimensional Clifford algebra by rescaling and relabelling of the operators; here we just relabel them obtaining 
   \begin{equation}
    \left\{c_{i},c_{j}\right\} = 2\delta_{ij}
    \label{eq:algebra_majorana}
   \end{equation}
   \item this algebra and the previous properties are invariant with respect to orthogonal transformations $Q\in\mathcal{O}(2N):\ c\rightarrow Qc$;
   \item since the linear relation between them and the Dirac fermionic operators is invertible, the algebras of observables generated by the two sets are identical: they are related by a simple change of basis and thus provide the same physical description of the system.
   \end{itemize}
   Performing the substitution into the Hamiltonian \eqref{eq:quadr_ham},
    \begin{align*}
     H &= a^+Ma^{ } = a^+Da^{ } + a^+M'a^{ }\notag\\
       &=\frac{1}{4}\left[\cancel{c_1Dc_1} + \cancel{c_2Dc_2} +ic_2Dc_1-ic_1Dc_2+(c_1+ic_2)M'(c_1-ic_2)\right] =  \notag\\
       &=\frac{1}{4}\left[ic_2Dc_1-ic_1Dc_2+c_1\mathcolor{blue}{M'}c_1 + c_2\mathcolor{blue}{M'}c_2 +ic_2\mathcolor{red}{M'}c_1-ic_1\mathcolor{red}{M'}c_2\right]=\notag \\
       &=\frac{1}{4}\bigg[ic_2\left(D+\mathcolor{red}{\frac{M'+(M')^T}{2}}\right)c_1-ic_1\left(D+\mathcolor{red}{\frac{M'+(M')^T}{2}}\right)c_2 +\notag\\
       &\ \ \ +c_1\mathcolor{blue}{\frac{M'-(M')^T}{2}}c_1+c_2\mathcolor{blue}{\frac{M'-(M')^T}{2}}c_2\bigg]\notag
    \end{align*}
    where \begin{itemize}
           \item $M$ has been separated into its diagonal part $D$ and the rest $\displaystyle M'$;
           \item $c_rDc_r = \sum_iD_{ii}c_{i,r}^2 = \sum_i D_{ii}$ is a constant and can be omitted;
           \item $\displaystyle c_rM'c_s = - c_s(M')^Tc_r = \frac{1}{2}c_rM'c_s - \frac{1}{2}c_s(M')^Tc_r$ since $M'$ has zero diagonal and different fermions anticommute.\nl
          \end{itemize}
    Finally, the general form for the Hamiltonian is, using that $M$ is an hermitian matrix:
    \begin{subequations}
%        \renewcommand{\theequation}{\theparentequation.\arabic{equation}}
%     \begin{equation}
    \label{eq:quadr_ham_majorana}
    \begin{align}
%      \left\{
%      \begin{aligned}
     & H \equiv icAc\\
     & A = \frac{1}{4}\left(
     \begin{array}{cc}
       -i\frac{M'-(M')^T}{2} & -D - \frac{M'+(M')^T}{2} \\
       + D + \frac{M'+(M')^T}{2} & -i \frac{M'-(M')^T}{2}
        \end{array}
        \right)
      =\frac{1}{4}\left( 
      \begin{array}{cc}
        \Im[M'] & -D-\Re[M']\\
        D+\Re[M'] & \Im[M']
        \end{array}
        \right)
%       \end{aligned}
%       \right .
%      \end{equation}
    \end{align}
    \end{subequations}
    and thus $A$ is a generic real antisymmetric matrix.\\[0.3cm] Performing the same programme for the Lindblad operators \eqref{eq:quadr_lindb},
    \begin{align*}
   L &= P a + Q a^+=\\
     &= (P+Q)c_1 +i(-P+Q)c_2\\
   L^+&= P^*a^++Q^*a=\\
     &= (P^*+Q^*)c_1+i(P^*-Q^*)c_2  
    \end{align*}  
   where $X^*$ is merely the complex (\ubar{not hermitian!}) conjugate of a complex matrix $X$. \\We recall that $P$ and $Q$ are rectangular matrices whose number of rows is the number of independent channels $L_{\alpha}$ and whose number of coloumns is the number of lattice sites (and hence of independent fermionic operators). From this, we get
  \begin{subequations}
% \begin{equation}
  \label{eq:quadr_lindb_majorana}
% \left\{ 
  \begin{align}
%  \begin{aligned}
      &L    \equiv Rc, \quad \quad \quad \quad  L^+  = R^*c,\\
      &L^+L = \sum_{\alpha} L^+_{\alpha}L^{ }_{\alpha} = c(R^*)^TRc = cR^+Rc \equiv cSc \\[0.3 cm]
      &R = \left(\begin{array}{cc} P+Q & -i(P-Q)\end{array}\right) \\
      &S =\left(\begin{array}{cc} (P+Q)^+(P+Q) & -i(P+Q)^+(P-Q)\\i(P-Q)^+(P+Q)&(P-Q)^+(P-Q) \end{array}\right)
%  \end{aligned}
  \end{align} 
% \right.
% \end{equation}
  \end{subequations}
  \nl Hence, the Majorana QME in the compact form reads
  \begin{equation}
   \label{eq:quadr_dyn_majorana}
   \dot{\ro}=\left[cAc,\ro\right]+cR^T\ro R^*c - \frac{1}{2}\left\{cSc,\ro\right\}
  \end{equation}
  \np
  
  \subsection{Equation of motion for the Covariance Matrix using the QME}
  The aim of this subsection is to derive a closed equation of motion for the two-point correlation function at equal time: \begin{equation*}
  \mean{c_kc_l}(t)=\Tr\left[c_kc_l\ro(t)\right]=\frac{1}{2}\mean{2\delta_{ij}}(t)+\frac{1}{2}\mean{[c_k,c_l]}(t)=\cancel{1}+\frac{1}{2}\mean{[c_k,c_l]}(t)
  \end{equation*}
  Since the symmetric part of the product $c_kc_l$ is trivial at all times due to the trace-preserving property of the time evolution, it carries no physical information. Let us then define the real and antisymmetric Covariance Matrix, which is the meaningful physical object:
  \begin{align}
   &\Gamma_{kl}=\frac{i}{2}\mean{\left[c_k,c_l\right]}(t) &  \longleftrightarrow & & \Gamma = \frac{i}{2}\Tr\left[\left(c\otimes c - (c\otimes c)^T\right)\ro(t)\right]
   \label{eq:def_covariance}
  \end{align}
  The $\Gamma$ matrix and the knowledge of its equation of motion are useful because:
  \begin{itemize}
   \item it allows an exact study of the relaxational dynamics towards the stationary state in the non-interacting case \cite{Eisert2010};
   \item it allows to discuss the extension of the theory of critical phenomena to the context of non-pure Non Equilibrium Steady States (NESS), since the correlation length can be analitically evaluated and its (possibly critical) dependence on some parameter that appears in \eqref{eq:quadr_dyn_majorana} can then be tested;
   \item it makes easier to identify pure NESSs, since its eigenvalues obey an inequality saturated only by pure states.\nl
  \end{itemize}
  In the spirit of the compact notation above, what we want to compute is
  \begin{align*}
   \dot{\Gamma} &=  \frac{i}{2}\Tr\left[\left(c\otimes c - (c\otimes c)^T\right)\mathcal{L}\left[\ro(t)\right]\right]=  \frac{i}{2}\Tr\left[\mathcal{L}^+\!\!\left[c\otimes c - (c\otimes c)^T\right]\ro(t)\right]
  \end{align*}
  A way to compute  $\mathcal{L}^+\!\!\left[c\otimes c\right]$ and $\mathcal{L}^+\!\!\left[ (c\otimes c)^T\right]$ that requires little effort consists in the following algorithm:
  \begin{enumerate}
   \item keep the matrix structure explicit, labelling only the first and last index, with $ \circ$ and $\bullet$ respectively; \ubar{these indices are never contracted}: if they appear in the middle of an expression, the index contraction of all the other operators/matrices starts before them and ends after them, if necessary;
%    \item make the order of contractions explicit using arrows, from left to right;
   \item start swapping fermionic operators in order to have $ \circ$ at the beginning and $\bullet$ at the end of the expression respectively; 
   \item after each swap the original term gets a -1 factor and a new term is generated, where the two operators are replaced by their anticommutator; since this is in turn 2 times a delta of the indices, its effect will be to replace the contracted index of the other operator with either $ \circ$ or $\bullet$; if it still has the wrong position (not the first/last), transpose the matrix that now carries this index.
  \end{enumerate}\nl
  The detailed calculations are:
   \begin{align*}
   \mathcal{L}^+\!\!\left[c^{ }_{\circ}c^{ }_{\bullet}\right] =& - \left[cAc, c^{ }_{\circ} c^{ }_{\bullet}\right] +cR^+c^{ }_{\circ} c^{ }_{\bullet}Rc-\frac{1}{2}\left\{cSc,c^{ }_{\circ}c^{ }_{\bullet}\right\}\\
   cAcc^{ }_{\circ} c^{ }_{\bullet} =&\mathcolor{blue}{-cAc^{ }_{\circ}cc^{ }_{\bullet}} + 2cA^{ }_{\circ} c^{ }_{\bullet} = \\ =&\mathcolor{blue}{+c^{ }_{\circ}cAc c^{ }_{\bullet}} \mathcolor{red}{+2cA^{ }_{\circ} c^{ }_{\bullet}} \mathcolor{blue}{-2 A^{ }_{\circ}cc^{ }_{\bullet}}=\\ 
   =&+c^{ }_{\circ}cAc c^{ }_{\bullet} \mathcolor{red}{+ 2A^T_{\circ}c c^{ }_{\bullet}}-2 A^{ }_{\circ}cc^{ }_{\bullet}\\
   =&+c^{ }_{\circ}cAc c^{ }_{\bullet} \mathcolor{red}{- 2A^{ }_{\circ}c c^{ }_{\bullet}}-2 A^{ }_{\circ}cc^{ }_{\bullet}\\
   c^{ }_{\circ} c^{ }_{\bullet} cAc =& \dots = +c^{ }_{\circ}cAc c^{ }_{\bullet}\mathcolor{cyan}{+2c^{ }_{\circ}c A^T_{\bullet}} -2c^{ }_{\circ}cA^{ }_{\bullet}= \\
   =& +c^{ }_{\circ}cAc c^{ }_{\bullet}\mathcolor{cyan}{-2c^{ }_{\circ}c A_{\bullet}} -2c^{ }_{\circ}cA^{ }_{\bullet}\\
   cR^+c^{ }_{\circ} c^{ }_{\bullet}Rc =& \mathcolor{blue}{-c^{ }_{\circ}cR^+c^{ }_{\bullet}Rc} \mathcolor{red}{+ 2R^+_{\circ}c^{ }_{\bullet}Rc} = \\
   =&\mathcolor{blue}{+c^{ }_{\circ}cR^+Rcc^{ }_{\bullet}-2c^{ }_{\circ}cR^+R^{ }_{\bullet}} \mathcolor{red}{-2R^+_{\circ}Rcc^{ }_{\bullet}+4R^+_{\circ}R^{ }_{\bullet}id}=\\
   =&+c^{ }_{\circ}cScc^{ }_{\bullet}-2c^{ }_{\circ}cS^{ }_{\bullet}-2S^{ }_{\circ}cc^{ }_{\bullet}+4S^{ }_{\circ\bullet}id \\[0.3cm]
   \mathcal{L}^+\!\!\left[c^{ }_{\circ}c^{ }_{\bullet}\right] =&
	     \mathcolor{green}{\cancel{+c^{ }_{\circ}cAc c^{ }_{\bullet}}} \mathcolor{blue}{-2c^{ }_{\circ}c A^{ }_{\bullet} -2c^{ }_{\circ}cA^{ }_{\bullet}}
	     \mathcolor{green}{\cancel{-c^{ }_{\circ}cAc c^{ }_{\bullet}}} \mathcolor{red}{+2A^{ }_{\circ}c c^{ }_{\bullet} +2A^{ }_{\circ}cc^{ }_{\bullet}} \\
	     &\cancel{+ c^{ }_{\circ}cScc^{ }_{\bullet}} \mathcolor{blue}{-2c^{ }_{\circ}cS^{ }_{\bullet}}\mathcolor{red}{-2S^{ }_{\circ}cc^{ }_{\bullet}}\mathcolor{cyan}{+4S^{ }_{\circ\bullet}id} \\
	     &\cancel{-\frac{1}{2}c^{ }_{\circ}cSc c^{ }_{\bullet}} \mathcolor{blue}{-\frac{2}{2}c^{ }_{\circ}c S^T_{\bullet}+\frac{2}{2}c^{ }_{\circ}cS^{ }_{\bullet}} \\
	     &\cancel{-\frac{1}{2}c^{ }_{\circ}cSc c^{ }_{\bullet}}  \mathcolor{red}{-\frac{2}{2}S^T_{\circ}c c^{ }_{\bullet} +\frac{2}{2}S^{ }_{\circ}cc^{ }_{\bullet}}=\\
	    =& \mathcolor{blue}{c^{ }_{\circ}c(-4A-S-S^T)^{ }_{\bullet}} + \mathcolor{red}{(+4A-S-S^T)^{ }_{\circ}cc^{ }_{\bullet}} + \mathcolor{cyan}{4S^{ }_{\circ\bullet}id}=\\
	    =& c^{ }_{\circ}c(-4A-2\Re[S])^{ }_{\bullet} + (+4A-2\Re[S])^{ }_{\circ}cc^{ }_{\bullet} + 4S^{ }_{\circ\bullet}id
   \end{align*} \nl
   The main result of this calculation is that all quartic terms in the equations of motion exactly cancel: the set of equations for $\{\mean{c_kc_l}(t)\}_{k,l}$ is closed. The full matricial form of these equations can be easily read from above, recalling that an ordered index contraction is omitted:
   \begin{subequations}
   \label{eq:covariance_qme}
   \begin{align}
    \tilde{\Gamma}_{kl}(t) \equiv & \{\mean{c_kc_l}(t)\}_{k,l},\quad\quad\quad\quad \Gamma = \frac{i}{2}\left(\tilde{\Gamma}-\tilde{\Gamma}^T\right) \notag\\
    \dot{\tilde{\Gamma}} =&\tilde{\Gamma}\big(\mathcolor{red}{-}4A\mathcolor{red}{-}2\Re[S]\big) + \mathcolor{blue}{\big(\mathcolor{red}{+}4A\mathcolor{red}{-}2\Re[S]\big)}\tilde{\Gamma} + 4S=\notag\\ &\mathcolor{red}{-}\left[\mathcolor{blue}{\big(4A\mathcolor{red}{+}2\Re[S]\big)^{T}}\tilde{\Gamma}+\tilde{\Gamma}\big(4A\mathcolor{red}{+}2\Re[S]\big)\right] +4S  \notag\\[0.3cm]
    \dot{\Gamma}=& -\left[\big(4A+2\Re[S]\big)^{T}\Gamma+\Gamma\big(4A+2\Re[S]\big)\right]+\mathcolor{cyan}{2i\big(S-S^T\big)}=\notag\\
                =& -\left[\big(4A+2\Re[S]\big)^{T}\Gamma+\Gamma\big(4A+2\Re[S]\big)\right]\mathcolor{cyan}{-4\Im[S]}=\label{eq:covariance_eisert}\\
                \equiv& -i\big[Z,\Gamma\big] - \big\{X,\Gamma\big\}+Y\label{eq:covariance_qme_compact}\\
    X =& S+S^T = 2\Re[S], \quad\quad\quad\quad (S\geq 0\Rightarrow X\geq 0) \notag\\
    Y =& 2i\big(S-S^T\big) = -4\Im[S] \notag \\
    Z =& 4iA \notag
   \end{align}
   \end{subequations}
   The result at step \eqref{eq:covariance_eisert} can be directly compared with the analogous equation in \cite{Eisert2010}, there are two differences: 
   \begin{itemize} 
   \item there's a (qualitatively irrelevant) factor of 2 instead of 4 multiplying A in the hamiltonian sector, it can be absorbed into a re-definition of the hamiltonian; 
   \item a crucial minus sign in front of the square bracket is missing; this cannot be re-absorbed since the matrix X is positive-definite for every choice of the Lindblad operators. Its physical meaning becomes clear when one writes the equation for the stationary covariance matrix $\Gamma_s$ and for $\delta\Gamma=\Gamma-\Gamma_s$:
   \begin{subequations}
    \begin{align}
     &\big\{ X,\Gamma_s\big\}+i\big[Z,\Gamma_s\big] = Y  \label{eq:covariance_qme_stationary} \\
     &\dot{\delta\Gamma}=-\big\{ X,\delta\Gamma\big\}-i\big[Z,\delta\Gamma\big] \label{eq:covariance_qme_approach}
    \end{align}
   \end{subequations}
   Due to the minus sign and the positive-definiteness of X, the Liouvillian dynamics towards the SS that solves \eqref{eq:covariance_qme_approach} is then necessarily relaxational: all the decay rates  must lie on the positive real-side of the complex plane (see \cite{Eisert2010} for a reference to the mathematical proof of this statement).\\ As pointed out in \cite{Eisert2010}, the same result doesn't hold for bosons, which can exhibit dynamical instabilities.\end{itemize}\np


   
 \section{Functional Integral reformulation of the QME}
 The rules of transcription from a QME to a functional integral for Dirac modes are known (from a preceding discussion?); the main question is to what extent similar rules can be derived for Majorana modes.\\ In the first part of this section, we derive them extending an easy and symmetric method developed in the Hamiltonian context (``fermion doubling'', \cite{Nilsson2013}), leaving some details to Appendix \eqref{sec:majorana_doubling_theory}. This allows to write a Majorana Keldysh action for the system. \\In the second part, the equation for the stationary Covariance Matrix is derived again using the Keldysh formalism.\np
  \subsection{From Majorana QME to Keldysh action}
  The main mathematical tool used to derive the Keldysh action for bosons or Dirac fermions directly from a QME is the coherent state decomposition of the identity, since it allows to ``trotterize'' the time evolution operator. This, together with time ordering and equal time regularization prescriptions, provides a fully self-consistent mapping from QME to Keldysh without the need to reformulate the problem in terms of a system+bath hamiltonian dynamics. However, this tool is not directly useful in our case, because Majorana operators don't have Grassmann coherent eigenstates, as aready pointed out.\\ The simplest strategy to circumvent the problem is to switch back to Dirac modes using \eqref{eq:def_majorana} (that is, to derive the Keldysh action directly from \eqref{eq:quadr_dyn}) and then to perform the Majorana rotation to the Grassmann fields. As pointed out in \cite{Nilsson2013} in the Hamiltonian scenario, this method treats $c_1$ and $c_2$ on a different footing, restoring the symmetry only at the end of the calculation. A symmetric way would be more satisfactory and turns out to lead to much easier calculations too.\\ To this end, let us double the dimension of the Hilbert space introducing fictious Majorana modes $d_j\,j\in\{1\dots2N\}$ and coupling both to form new Dirac modes
  \begin{equation}
   \label{eq:def_majorana_doubling}
   \begin{aligned}
   &c_j^{ } = b_j^{ }+b_j^+\\
   &d_j^{ } = i\left(b^{ }_j - b^{+}_j\right)
   \end{aligned}
  \end{equation}
  the inverse being the same as in \eqref{eq:def_majorana}. Since fictious Majorana modes are decoupled from dynamics, if the initial state has zero projection on the unphysical subspace its dynamical evolution will be the same as before. The correlation functions of the physical modes are then unchanged.\\ The programme is thus: perform the first substitution of \eqref{eq:def_majorana_doubling} into the QME, use the known rules to write a Keldysh action and a generating functional for the Dirac modes, rotate back to Majorana modees and integrate out the $d$ field (since it doesn't enter in any physical correlator). This integration will be very simple because the fictious field appears only in the quadratic part containing time derivatives.\\[0.3cm] Details (definition of the physical subspace, precise statement about dynamical evolution being ``the same as before'', boundary conditions to be imposed on the d field) are carried out in the already mentioned Appendix \eqref{sec:majorana_doubling_theory}. It's only important for now to remark that the customary convention for Keldysh rotation for fermions (see \cite{Kamenev2011} for the original reference) is not the best choice in our case: since Grassmann integrals are always convergent, different linear combinations of $\left\{\psi^{\pm}\right\}$ and $\left\{\psibar^{\pm}\right\}$ are usually chosen to form respectively $\left\{\psi_1,\psi_2\right\}$ and $\left\{\psibar_1,\psibar_2\right\}$, with the advantage of simplifying greatly calculations; however, since $\psi$ and $\psibar$ are not treated on an equal footing anymore, this rotation in Keldysh space does not commute with Majorana rotation in Nambu space. In formulae:
  \begin{align*}
  &\begin{aligned}
   \begin{pmatrix} 
    \psi^+ \\ \psi^- \\ \psibar^+ \\ \psibar^-
   \end{pmatrix}
   & \ra 
   \begin{pmatrix}
   \psi_1\\ \psi_2 \\ \psibar_1 \\ \psibar_2 
   \end{pmatrix}
   = \frac{1}{\sqrt{2}}
   \begin{pmatrix}
   1 & 1 & 0 & 0\\ 1 & -1 & 0 & 0 \\ 0 & 0 & 1 & 1\\ 0 & 0 & -1 & 1 
   \end{pmatrix}
   \begin{pmatrix} 
    \psi^+ \\ \psi^- \\ \psibar^+ \\ \psibar^-
   \end{pmatrix}
   \ra
   \mathcolor{red}{\begin{pmatrix}
   c_{1,1}\\ c_{2,1} \\ c_{1,2} \\ c_{2,2} 
   \end{pmatrix}}
   = \frac{1}{\sqrt{2}}
   \begin{pmatrix}
   1 & 0 & 1 & 0\\ 0 & 1 & 0 & 1 \\  i & 0 & -i & 0\\ 0 & i & 0 & -i 
   \end{pmatrix}\\
   &\begin{pmatrix}
   1 & 1 & 0 & 0\\ 1 & -1 & 0 & 0 \\ 0 & 0 & 1 & 1\\ 0 & 0 & -1 & 1 
   \end{pmatrix}
   \begin{pmatrix} 
    \psi^+ \\ \psi^- \\ \psibar^+ \\ \psibar^-
   \end{pmatrix}
    = \frac{1}{\sqrt{2}}
   \mathcolor{red}{\begin{pmatrix}
   1 & 1 & 1 & 1\\ 1 & -1 & -1 & 1 \\ i & i & -i & -i\\ i & -i & i & -i
   \end{pmatrix}}
   \begin{pmatrix} 
   \psi^+ \\ \psi^- \\ \psibar^+ \\ \psibar^-
   \end{pmatrix}
   \end{aligned} \\
  &\begin{aligned}
    \begin{pmatrix} 
    \psi^+ \\ \psi^- \\ \psibar^+ \\ \psibar^-
    \end{pmatrix}
    & \ra 
    \begin{pmatrix} 
    c_1^+ \\ c_1^- \\ c_2^+ \\ c_2^-
    \end{pmatrix}
    =
    \begin{pmatrix}
   1 & 0 & 1 & 0\\ 0 & 1 & 0 & 1 \\  i & 0 & -i & 0\\ 0 & i & 0 & -i 
   \end{pmatrix}
   \begin{pmatrix} 
    \psi^+ \\ \psi^- \\ \psibar^+ \\ \psibar^-
   \end{pmatrix}
   \ra
   \mathcolor{blue}{\begin{pmatrix}
   c'_{1,1}\\ c'_{2,1} \\ c'_{1,2} \\ c'_{2,2} 
   \end{pmatrix}}
   = \frac{1}{\sqrt{2}}
   \begin{pmatrix}
   1 & 1 & 0 & 0\\ 1 & -1 & 0 & 0 \\ 0 & 0 & 1 & 1\\ 0 & 0 & -1 & 1 
   \end{pmatrix}\\
   & \begin{pmatrix}
   1 & 0 & 1 & 0\\ 0 & 1 & 0 & 1 \\  i & 0 & -i & 0\\ 0 & i & 0 & -i 
   \end{pmatrix}
   \begin{pmatrix} 
    \psi^+ \\ \psi^- \\ \psibar^+ \\ \psibar^-
   \end{pmatrix}
    = \frac{1}{\sqrt{2}}
   \mathcolor{blue}{\begin{pmatrix}
   1 & 1 & 1 & 1\\ 1 & -1 & 1 & -1 \\ i & i & -i & -i\\ -i & i & i & -i
   \end{pmatrix}}
   \begin{pmatrix} 
   \psi^+ \\ \psi^- \\ \psibar^+ \\ \psibar^-
   \end{pmatrix}
   \end{aligned}
  \end{align*}
  To avoid this ambiguity, we shall adopt a different convention: just like the bosonic case, $\psi$ and $\psibar$ will be rotated with the same matrix, so that the block matrix above has two equal diagonal blocks. This will ensure Keldysh Majorana fields to be unambiguously defined:
  \begin{align*}
   \begin{pmatrix}
   1 & 0 & 1 & 0\\ 0 & 1 & 0 & 1 \\  i & 0 & -i & 0\\ 0 & i & 0 & -i 
   \end{pmatrix}&
   \begin{pmatrix}
   1 & 1 & 0 & 0\\ 1 & -1 & 0 & 0 \\ 0 & 0 & 1 & 1\\ 0 & 0 & 1 & -1 
   \end{pmatrix}
   =
   \begin{pmatrix}
   1 & 1 & 0 & 0\\ 1 & -1 & 0 & 0 \\ 0 & 0 & 1 & 1\\ 0 & 0 & 1 & -1 
   \end{pmatrix}
   \begin{pmatrix}
   1 & 0 & 1 & 0\\ 0 & 1 & 0 & 1 \\  i & 0 & -i & 0\\ 0 & i & 0 & -i 
   \end{pmatrix}=\\
  &= \begin{pmatrix}
   1 & 1 & 1 & 1\\ 1 & -1 & 1 & -1 \\ i & i & -i & -i\\ i & -i & -i & i
   \end{pmatrix}
  \end{align*}
  The detailed calculations follow:
  \begin{align*}
   &H=icAc=i(b+b^+)A(b+b^+),\quad \quad \quad \quad L=Rc=R(b+b^+) \\
   &Z[\eta_+,\obar{\eta}_+,\eta_-,\obar{\eta}_-] = \int D[b_+,\obar{b}_+]D[b_-,\obar{b}_-]\,\,e^{iS+\int\obar{b}_+\eta_+-\obar{b}_-\eta_-+\obar{\eta}_+ b_++\obar{\eta}_- b_- }\\
   &S=\int_t\obar{b}_+i\partial_tb_+-\obar{b}_-i\partial_tb_--i(b_++\obar{b}_+)A(b_++\obar{b}_+)+i(b_-+\obar{b}_-)A(b_-+\obar{b}_-)\\
   &\quad -i\left((b_-+\obar{b}_-)S(b_++\obar{b}_+)-\frac{1}{2}(b_++\obar{b}_+)S(b_++\obar{b}_+)-\frac{1}{2}(b_-+\obar{b}_-)S(b_-+\obar{b}_-) \right)\\
   \end{align*}
   As expected, the hamiltonian and Lindbladian part do not depend on the fictious field $d$. Performing the Majorana rotation,
   \begin{align*}
   &\begin{pmatrix}
   b_+ \\ \obar{b}_+ \\ b_- \\ \obar{b}_-            
   \end{pmatrix}
   \ra
   \begin{pmatrix}
   c_+ \\ d_+ \\ c_- \\ d_-
   \end{pmatrix}
   = 
   \begin{pmatrix}
   1 & 1 & 0 & 0\\ i & -i & 0 & 0 \\ 0 & 0 & 1 & 1\\ 0 & 0 & i & -i 
   \end{pmatrix}
   \begin{pmatrix}
   b_+ \\ \obar{b}_+ \\ b_- \\ \obar{b}_-            
   \end{pmatrix}\\
   &Z\ra \prod_{j=1}^{4\cdot2N}\left(\frac{i}{2}\right)^{-1} \cdot \int D[c_+,d_+]D[c_-,d_-]\,\,e^{iS'+\int \lambda_+c_+-\lambda_-c_-+\chi_+d_+-\chi_-d_- }\quad\text{\footnotemark }\\
   &S'=\int_t \frac{1}{4}\big[c_+i\partial_tc_+-c_-i\partial_tc_- + d_+i\partial_td_+-d_-i\partial_td_-+\mathcolor{blue}{c_+\partial_td_+-d_+\partial_t c_+ - c_-\partial_td_--d_-\partial_t c_-}\big]\\&\quad +\tilde{S}[c_+,c_-]=\\
   &\quad =\int_t \frac{1}{4}\big[c_+i\partial_tc_+-c_-i\partial_tc_- + d_+i\partial_td_+-d_-i\partial_td_-+\mathcolor{blue}{\partial_t(c_+d_+-c_-d_-)}\big]+\tilde{S}[c_+,c_-]
   \end{align*} \footnotetext{The correct normalization comes from Berezin's formula, look for references! 2N = number of physical 
   Majorana modes = number of fictious modes, \ubar{for each branch of the contour}.}
   
   The total derivative contributes with boundary terms wihich sum up to 0: $c_+=c_-$ and $d_+=d_-$ at the final time by construction of the functional integral; since the initial state has no projection onto the subspace where $d$ has support, the initial value of this field is 0.\\
   The integration upon $d$ is then trivial: it is a fermionic Gaussian one, starting on the + contour and continuing on the reversed -contour, with $d(t_i)=d(t_f)=d_{\pm}(t_i)=0$ ; ignoring the sources (they can be absorbed by a shift of the integration variables), it reads 
   \begin{align*}
   &\int D[d]\,\exp\left[\int_t \frac{1}{2}dMd\right] = \Det\left[\frac{M-M^T}{2}\right]^{1/2}, & &M=-\frac{1}{2}\partial_t
   \end{align*}
   and, in a discrete representation,
   \begin{align*}
    &M = \frac{1}{2}\begin{pmatrix}
                     1 & 0 & \hdots & \mathcolor{red}{0} \\ -1 & 1 & 0 &\hdots \\ 0 & -1 & 1 & \ddots \\ \vdots & \ddots  & \ddots & \ddots 
                    \end{pmatrix}, 
    & & \frac{M-M^T}{2} = \frac{1}{4}\begin{pmatrix}
                     0 & 1 & \hdots & \mathcolor{red}{0} \\ -1 & 0 & 1 &\hdots \\ 0 & -1 & 0 & \ddots \\ \vdots & \ddots  & \ddots & \ddots 
                    \end{pmatrix}\\
    & \Det\left[\frac{M-M^T}{2}\right]^{1/2} = \left[4^{-2\cdot2N}\right]^{1/2} = 2^{-2\cdot2N}
   \end{align*}
   The net effect of the $d$ field is thus only cancelling half of the normalization factor (as it should be, since if $\tilde{S}=0$,  $c$ and $d$ equally contribute to $Z=1$). \\[0.3cm] This result can be summed up in the following rule:  \begin{framed}The Keldysh action for Majorana fermions can be written using the same rules of Dirac fermions, with an additional factor of 1/2 (or 1/4 + normalization prefactor to the functional integral) multiplying the time derivative term, depending on the normalization chosen for Majorana operators.\end{framed} Therefore, the action is 
   \begin{align}
    S =& \int_t \left[\frac{1}{4}c_+i\partial_tc_+-\frac{1}{4}c_-i\partial_tc_- -ic_+Ac_+ + ic_-Ac_- -i \left(c_-Sc_+ -\frac{1}{2}c_+Sc_+ -\frac{1}{2}c_-Sc_-\right)\right]= \notag\\ =& \int_t \left[ \frac{1}{4} c_1\left(i\partial_t -4iA\right)c_2 + \frac{1}{4} c_2\left(i\partial_t -4iA\right)c_1 -\frac{i}{2}(c_1Sc_2-c_2Sc_1-2c_2Sc_2) \right]= \notag \\
    =&\ \frac{1}{4} \int_t 
               \begin{pmatrix}
                c_1 & c_2
               \end{pmatrix}
               \begin{pmatrix}
                0 & i\partial_t - Z - iX \\ i\partial_t - Z + iX & Y
               \end{pmatrix}
               \begin{pmatrix}
                c_1 \\ c_2
               \end{pmatrix}\\[0.3cm]
    X =&\  S+S^T = 2\Re[S], \quad\quad\quad\quad (S\geq 0\Rightarrow X\geq 0) \notag\\
    Y =&\ 2i\big(S-S^T\big) = -4\Im[S] \notag \\
    Z =&\ 4iA \notag 
   \end{align}
   where the inverse Green functions matrix has been antisymmetrized without loss of generality (except the time derivative part) and the definitions after \eqref{eq:covariance_qme} appear again in a natural way.
  \np
  \subsection{Equation of motion for the Covariance Matrix using Kelidysh}
  In order to relate the equal time 2-point correlation function to Green functions in the functional integral framework, some care has to be taken: ``equal time'' is ill-defined in the continuum limit and a time ordering prescription is necessary, either discretizing time or insterting it directly into the continuum form. \\ The former will always produce correct results, so it provides us an insight for the correct choice for the latter too:
  \begin{align*}
  \mathcolor{red}{\mean{\hat{c_i}\hat{c_j}}(t)}&=\Tr\left[ \underset{\mathclap{\longleftarrow}}{\hat{c_i}\hat{c_j}}\hat{\rho}(t)\right] = \lim_{\delta\to 0^+}\mathcolor{red}{\mean{c_{+i}(t+\delta)c_{+j}(t)}}=\notag\\ 
  &=\Tr\left[ \underset{\mathclap{\leftarrow}}{\hat{c_j}}\hat{\rho}(t)\underset{\mathclap{\rightarrow}}{\hat{c_i}}\right] = \lim_{\delta,\delta'\to 0} \mean{c_{-i}(t+\delta)c_{+j}(t+\delta')} = \lim_{\delta\to 0} \mathcolor{red}{\mean{c_{-i}(t+\delta)c_{+j}(t)}}  \notag\\
  \mathcolor{blue}{\mean{\hat{c_j}\hat{c_i}}(t)}&=\Tr\left[\hat{\rho}(t) \underset{\mathclap{\longrightarrow}}{\hat{c_j}\hat{c_i}}\right] = \lim_{\delta\to 0^+}\mean{c_{-j}(t)c_{-i}(t+\delta)}=-\lim_{\delta\to 0^+}\mathcolor{blue}{\mean{c_{-i}(t+\delta)c_{-j}(t)}}=\notag\\
   &=\Tr\left[ \underset{\mathclap{\leftarrow}}{\hat{c_i}}\hat{\rho}(t)\underset{\mathclap{\rightarrow}}{\hat{c_j}}\right]= \lim_{\delta,\delta'\to 0} \mean{c_{-j}(t+\delta')c_{+i}(t+\delta)}=-\lim_{\delta\to 0} \mathcolor{blue}{\mean{c_{+i}(t+\delta)c_{-j}(t)}} \notag\\[0.3cm]
  \mean{c_{1i}(t+\delta)c_{1j}(t)} &= \frac{1}{2}(\mathcolor{red}{\mean{c_{+i}(t+\delta)c_{+j}(t)}}+\mathcolor{blue}{\mean{c_{+i}(t+\delta)c_{-j}(t)}}+\mathcolor{red}{\mean{c_{-i}(t+\delta)c_{+j}(t)}}\notag \\ & \quad +\mathcolor{blue}{\mean{c_{-i}(t+\delta)c_{-j}(t)}})\underset{\mathclap{\delta\to0^+}}{\longrightarrow}\frac{1}{2}(\mathcolor{red}{2\mean{\hat{c_i}\hat{c_j}}(t)}-\mathcolor{blue}{2\mean{\hat{c_j}\hat{c_i}}(t)})=\mean{\left[\hat{c_i},\hat{c_j}\right]}(t)\notag\\
  \end{align*}
  \begin{equation}
  \Gamma_{ij}=\frac{i}{2}\lim_{\delta\to0^+}\mean{c_{1i}(t+\delta)c_{1j}(t)}=\frac{i}{2}\lim_{\delta\to0^+}iG^K_{ij}(t+\delta,t)=-\frac{1}{2}\lim_{\delta\to0^+}G^K_{ij}(t+\delta,t)
  \end{equation}
  Analogously, it can be proved in the same way
  \begin{equation*}
   \Gamma_{ij}=-\frac{1}{2}\lim_{\delta\to0^+}G^K_{ij}(t,t+\delta)
  \end{equation*}
  \nl
  The equation of motion $G_K$ obeys can be derived from the action: $G_K=-G^RP^KG^A$ and, in our case \footnote{The factor 2 comes from the normalization chosen: a factor $1/2$ in front of the action is the ``standard'' normalization and every excess contributes to Green functions. },
  \begin{equation*}
   \begin{array}{ccc}
    G^R=2\frac{1}{i\partial_t-Z+iX} & G^A=2\frac{1}{i\partial_t-Z-iX} & P^K=\frac{1}{2}Y
   \end{array}
   \end{equation*}
  \begin{equation}
   (G^R)^{-1}G^K = -P^KG^A \quad\ra\quad (i\partial_t -Z -iX)G^K(t,t')=-2YG^A(t,t') \label{eq:keldysh_dysoneq}
  \end{equation}
  Now we can derive the equation of motion $G^K(t+\delta,t)$ obeys. We divide the derivation into steps: 
  \begin{enumerate}
   \item for $t\neq t'$, $\mean{c_i(t)c_j(t')}=-\mean{c_j(t')c_i{t}} \quad \ra \quad G^K(t',t)=-[G^K(t,t')]^T$
   \item from \eqref{eq:keldysh_dysoneq}, $$\displaystyle i\partial_tG^K(t+\delta,t')|_{t'=t} = ZG^K(t+\delta,t) -iXG^K(t+\delta,t)-YG^A(t+\delta,t)$$
   \item $\displaystyle i\partial_{t'}G^K(t+\delta,t')|_{t'=t} = -\left[i\partial_{t'}G^K(t',t+\delta)\right]^T$, so, from \eqref{eq:keldysh_dysoneq} again and using $\displaystyle X^T=X,\ Y^T=-Y,\ Z^T=-Z$, one has $$\displaystyle i\partial_{t'}G^K(t+\delta,t')|_{t'=t} = - G^K(t+\delta,t)Z - G^K(t+\delta,t)iX-\left[G^A(t,t+\delta)\right]^TY$$
   \item $\displaystyle G^A_{jk}(t,t')=+i\theta(t'-t)\left\{\hat{c_j}(t),\hat{c_k}(t')\right\}$, so $\displaystyle G^A_{jk}(t+0^+,t)=0,\ G^A_{jk}(t,t+0^+)=2i\delta_{jk}$
   \item $\displaystyle\dot{\Gamma}=-\frac{1}{2}\lim_{\delta\to0^+}\frac{d}{dt}G^K(t+\delta,t)=-\frac{1}{2}\lim_{\delta\to0^+}\left[\partial_tG^K(t+\delta,t')|_{t'=t}+\partial_{t'}G^K(t+\delta,t')|_{t'=t}\right]$; at the end we get
   \begin{align*}
    i\dot{\Gamma} &= Z\Gamma - \Gamma Z - iX\Gamma -i\Gamma X +\frac{1}{2}\lim_{\delta\to 0^+}\left[\cancel{YG^A(t+\delta,t)}+ G^A(t+\delta,t)^TY\right] = \\
    &= \big[Z,\Gamma\big] -i\big\{X,\Gamma\big\} +iY \\
    \dot{\Gamma} &= -i\big[Z,\Gamma\big] -\big\{X,\Gamma\big\} +Y
   \end{align*}
   which is the desired result.
  \end{enumerate}
 \pagebreak

 \appendix
 \section{Majorana doubling trick}
 \label{sec:majorana_doubling_theory}
 Plan of the Appendix (not going to write it explicitly until the very end, unless you want to see it before):
 \begin{itemize}
  \item uniqueness of real representations of Grassmann algebra (cite \cite{Okubo1991real}) $\ra$ whatever the original basis may be, there exists a change of basis that reduces Majorana operators matrices as X111, ZX11, ZZX1, ...
  \item given this basis, complexification and mapping into Dirac algebra is trivial since these are the matrix elements in the Dirac basis; complexification transforms operators in this way: $f(x) \ra \tilde{f}(v)=f(x)\oplus if(y)$; in order to keep observables well defined this map has to be enforced: the physical subspace contains only vectors whose coefficients wrt this basis have equal real and imaginary part, in this way the restriction of all observables is again a well defined real linear function; call P the projector onto this subspace; 
  \item cP=Pc, while PdP=0; due to this, $\mathcal{L}[P\cdot]=P\mathcal{L}[\cdot]$, $\mathcal{L}[\cdot P]=\mathcal{L}[\cdot]P$;
  \item consider a density matrix with initial support on the physical subspace, $P\rho P = \rho$; due to the property of the Liouvillian dynamics above, this property is preserved at all times $\ra$ the density matrix evolves only within the physical subspace with the same Liouvillian as before $\ra$ it is the same as before, same matrix elements within the p.s. and 0 every other
  \item correlation functions of the physical operators are unchanged: P and $c(t)$ commute at all times; the Heisenberg operator $PcP(t)$ is the same (the same matrix elements on the p.s. and 0 every other) as $c(t)$ before; $\mean{c(t)c(t')\dots}$ before = $\mean{c(t) c(t')\dots}$ now (\textit{proof:} $\mean{c(t)c(t')\dots}$ before = $\mean{Pc(t)P\, Pc(t')P\dots}$ is true because they have the same matrix elements; $P\rho P = \rho$, so in the correlator on the extended space the initial state can ``produce'' as many P as required; P and $c(t)$ commute at all times, so one can reproduce the correlator with all the P insertions);
  \item usual functional-integral-from-QME derivation now perfectly works: we have a complex hilbert space, Dirac modes and Grassmann variables. Boundary conditions are the last problem: it must be proven that the d field is 0 at the initial time. Possible proof: expand $\rho$ in terms of all the observables generated by Majorana modes (possible because they produce the same algebra of observables of Dirac modes!) in order to easily enforce $P\rho P=\rho$; write the obtained expansion in terms of Dirac modes and compute the matrix elements; switch back to Grassmann fields $c$ and $d$ to check boundary conditions.
 \end{itemize}
 \pagebreak

\bibliography{../library}{}
\bibliographystyle{alpha}  
\end{document}
