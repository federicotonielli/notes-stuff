\documentclass[a4paper,11pt]{article}

\usepackage{amsmath, amssymb, amsthm}
\usepackage{mathtools}
\usepackage{vmargin}

\usepackage[all]{xy}
\usepackage{graphicx}

\usepackage[utf8]{inputenc}
\usepackage[T1]{fontenc}   
\usepackage[english]{babel}
% \usepackage{fullpage}
\usepackage{hyperref}

\usepackage{enumitem}
\usepackage{xfrac}
\usepackage{slashed}
\usepackage{cancel}

\usepackage{afterpage}

\newcommand\blankpage{%
    \null
    \thispagestyle{empty}%
    \addtocounter{page}{-1}%
    \newpage}

%-------------------------------------------------------------------------------

\allowdisplaybreaks[4]

%-------------------------------------------------------------------------------

\newcommand{\ssection}[2]{\section{ \texorpdfstring{\textbf{#1}}{#2} }}

\newcommand{\HRule}{\center{\rule{0.9\linewidth}{0.5mm}}}

% add these definistions to the definitions' file!!!!!!! --------------------------------------

\DeclareMathOperator{\sign}{sign}
\DeclareMathOperator{\Tr}{Tr}
\DeclareMathOperator{\tr}{tr}

\newcommand{\mean}[1]{\ensuremath{\langle #1 \rangle}}
\newcommand{\kett}[1]{\ensuremath{\left | #1 \right \rangle}}
\newcommand{\brat}[1]{\ensuremath{\left \langle  #1 \right | }}
\newcommand{\ket}[1]{\left | #1 \right \rangle}
\newcommand{\bra}[1]{\left \langle  #1 \right |}
\newcommand{\ro}{\rho}

\newcommand{\nl}{\vskip 0.3cm}
\newcommand{\np}{\vskip 1.3cm}

\newcommand{\obar}[1]{\overline{#1}}
\newcommand{\ubar}[1]{\underline{#1}}
\newcommand{\psibar}{\bar{\psi}}
\newcommand{\sigmatilde}{\tilde{\sigma}}

\newcommand{\der}[2]{\frac{\partial #1}{\partial #2}}
\newcommand{\demu}[1]{\partial#1{\mu}}
\newcommand{\denu}[1]{\partial#1{\nu}}
\newcommand{\pder}[3]{\dfrac{\partial^{#1} #2}{\partial^{#1} #3}}
\newcommand{\de}[1]{\frac{d^2#1}{(2\pi)^2}}
 
% ---------------------------------------------------------------------------------------------


% ---- info ---------
\title{Notes: Fermionic model of interest}
\author{Federico Tonielli}

% ---- document -----
\begin{document}

 \maketitle
 
 \tableofcontents
 \blankpage
 
 \section{The second-quantized model}
 Here we present the general model under study in a second quantized formalism.  Moreover, the full equation of motion of the two-point correlation matrix at equal times is derived.
 
  \subsection{Quantum Master Equation in Majorana form}
   We consider an open quantum system of Dirac fermions on a lattice, which we will consider finite for simplicity; the creation and annihilation operators for these fermions are $\left \{ a_i^{ },a_i^+, i\in \{1\dots N\} \right \} $\footnote{ The usual hat to distinguish between operators and numbers will always be omitted, as numbers (matrix elements) will be referred to using capital letters.} and they obey the usual anticommutation relations $\left\{a_i^{ },a_j^{ }\right\} = \left\{a_i^{+},a_j^+\right \}=0$ and $\left\{a_i^{ },a_j^+\right \} = \delta_{ij}\,id $.\\  We assume its dynamics to be described by a Markovian Quantum Master Equation (herein after referred as QME), which in the Lindblad form reads
  \[\dot{\ro}=\mathcal{L}[\ro]=-i\left[H,\ro\right]+\sum_{\alpha}\left[L_{\alpha}^{ }\ro L_{\alpha}^+ - \frac{1}{2}L_{\alpha}^+L_{\alpha}^{ }\ro - \frac{1}{2}\ro L_{\alpha}^+L_{\alpha}^{ }\right]  \]
  Furthermore, we will assume the Hamiltonian to be quadratic and the Lindblad operators to be linear in the fermionic fields  and , without any other constraint: 
  \begin{subequations}        %sub-numbering of the equations which follow
%   \renewcommand{\theequation}{\theparentequation.\arabic{equation}}         %sub-numbering is 1.1, 1.2 instead of standard 1a, 1b
  \label{eq:quadr_dyn}         %ref the global set of equations as (1) with \ref{eq:global_label}
  \begin{align}
   H & = a_j^+M_{jk}a_k^{ } \equiv a^+Ma \label{eq:quadr_ham}\\         %ref the first eq as (1a) with \ref{eq:local_1st_label}
   L_{\alpha} & = P^{ }_{\alpha j}a_j^{ } + Q^{ }_{\alpha j}a_j^+ \Rightarrow L \equiv P a + Q a^+ \label{eq:quadr_lindb}
  \end{align}
  \end{subequations}
 where a compact notation has been introduced in order to simplify the form of next calculations: 
 \begin{itemize}
  \item indices are omitted and the order of symbols in the expressions also indicates the order of index contraction (e.g. $Mc\rightarrow M_{ij}a_j$ and $cM \rightarrow a_j M_{ji}$);
  \item operators and matrices of numbers appear on the same footing in the formulae, according to the already mentioned rules of notations; there will be no need to distinguish between index-carrying and index-free operators/numbers as it will be always clear from the context.
 \end{itemize}
  \np
 Following \cite{Eisert2010}, it is convenient to describe the system using Majorana fermionic operators
 \begin{align}
  c_{j, 1} = a_j^{ }+a_j^{+} \quad \quad \quad
  c_{j, 2} = i(a_j^{ }-a_j^{+})
  \label{eq:def_majorana}
 \end{align}
  whose algebra is 
 \begin{align}
  \left\{c_{i,r},c_{j,s}\right\} = 2\delta_{ij}\delta_{rs}
  \label{eq:algebra_majorana}
 \end{align}
  as can be proven by direct computation. To what concerns us, they have these important properties:
  \begin{itemize}
   \item they are self-adjoint;
   \item they square to $id$, so they have no vacuum nor any Grassmann coherent state as an eigenstate;
   \item their algebra can be straightforwardly mapped into a 2N-dimensional Clifford algebra by rescaling and relabelling of the operators; 
   \item this algebra and the previous properties are invariant with respect to orthogonal transformations $Q\in\mathcal{O}(2N):\ c_i\rightarrow Qc$;
   \item since the linear relation between them and the Dirac fermionic operators is invertible, the algebras of observables generated by the two sets are identical: they provide the same physical description of the system. The 'switch' from Dirac to Majorana is thus a simple change of basis.
  \end{itemize}
  We'll perform this substitution into the Hamiltonian \ref{eq:quadr_ham}:
    \begin{align*}
     H &= a^+Ma^{ } = a^+Da^{ } + a^+M'a^{ }\notag\\
       &=\frac{1}{4}\left[\cancel{c_1Dc_1} + \cancel{c_2Dc_2} +ic_2Dc_1-ic_1Dc_2+(c_1+ic_2)M'(c_1-ic_2)\right] =  \notag\\
       &=\frac{1}{4}\left[ic_2Dc_1-ic_1Dc_2+c_1M'c_1 + c_2M'c_2 +ic_2M'c_1-ic_1M'c_2\right]=\notag \\
       &=\frac{1}{4}\bigg[ic_2\left(D+\frac{M'+(M')^T}{2}\right)c_1-ic_1\left(D+\frac{M'+(M')^T}{2}\right)c_2 +\notag\\
       &\ \ \ +c_1\frac{M'-(M')^T}{2}c_1+c_2\frac{M'-(M')^T}{2}c_2\bigg]\notag
   \end{align*}
   where $M$ has been separated into its diagonal part $D$ and the rest $\displaystyle M'$, $c_rDc_r = \sum_iD_{ii}c_{i,r}^2 = \sum_i D_{ii}$ is a constant and can be omitted, $\displaystyle c_rM'c_s = - c_s(M')^Tc_r = \frac{1}{2}c_rM'c_s - \frac{1}{2}c_s(M')^Tc_r$ since $M'$ has zero diagonal and different fermions anticommute.\\ Finally, the general form for the Hamiltonian is, using that $M$ is an hermitian matrix:
    \begin{subequations}
%        \renewcommand{\theequation}{\theparentequation.\arabic{equation}}
%     \begin{equation}
    \label{eq:quadr_ham_majorana}
    \begin{align}
%      \left\{
%      \begin{aligned}
     & H \equiv icAc\\
     & A = \left(
     \begin{array}{cc}
       -i\frac{M'-(M')^T}{2} & -D - \frac{M'+(M')^T}{2} \\
       + D + \frac{M'+(M')^T}{2} & -i \frac{M'-(M')^T}{2}
        \end{array}
        \right)
      =\left( 
      \begin{array}{cc}
        \Im[M'] & -D-\Re[M']\\
        D+\Re[M'] & \Im[M']
        \end{array}
        \right)
%       \end{aligned}
%       \right .
%      \end{equation}
    \end{align}
\end{subequations}
and thus $A$ is a generic real antisymmetric matrix.\\[0.3cm] Now we'll do the same for the Lindblad operators \ref{eq:quadr_lindb}:
\begin{align*}
 L &= P a + Q a^+=\\
   &= (P+Q)c_1 +i(-P+Q)c_2\\
 L^+&= P^*a^++Q^*a=\\
   &= (P^*+Q^*)c_1+i(P^*-Q^*)c_2  
\end{align*} 
where $X^*$ is merely the complex (\ubar{not hermitian!}) conjugate of a complex matrix $X$. \\We recall that $P$ and $Q$ are rectangular matrices whose number of rows is the number of independent channels $L_{alpha}$ and whose number of coloumns is the number of lattice sites (and hence of the independent fermionic operators). From this, we get
\begin{subequations}
% \begin{equation}
\label{eq:quadr_lindb_majorana}
% \left\{ 
\begin{align}
%  \begin{aligned}
    &L    \equiv Rc, \quad \quad \quad \quad  L^+  = R^*c,\\
    &R = \left(\begin{array}{cc} P+Q & -iP+iQ\end{array}\right) \\[0.3cm]
    &L^+L = cR^+Rc \equiv cSc \\
    &S =\left(\begin{array}{cc} (P+Q)^+(P+Q) & -i(P+Q)^+(P-Q)\\i(P-Q)^+(P+Q)&(P-Q)^+(P-Q) \end{array}\right)
%  \end{aligned}
\end{align} 
% \right.
% \end{equation}
\end{subequations}
  

  
  
  \subsection{Equation of motion for the Covariance Matrix from QME}
 \section{Path Integral reformulation of the QME}
  \subsection{Keldysh action for Majorana fermions}
  \subsection{Re-derivating the equation for the stationary Covariance Matrix}

  \appendix
  \section*{Appendix}
    \renewcommand{\thesection}{A}
    \subsection{``Majorana doubling'' trick}
    \label{sec:majo_doubling_theory}
    
   
   
\bibliography{../library}{}
\bibliographystyle{amsalpha}  
\end{document}
