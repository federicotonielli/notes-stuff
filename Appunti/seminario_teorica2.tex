\documentclass[a4paper,11pt]{article}
\usepackage[utf8x]{inputenc}


\usepackage{amsmath, amssymb, amsthm}
\usepackage{mathtools}
\usepackage{bbold}

\usepackage[all]{xy}
\usepackage{graphicx}

\usepackage{fullpage}
\usepackage{hyperref}
\usepackage[utf8x]{inputenc}
\usepackage[T1]{fontenc}   
\usepackage[italian]{babel}

\usepackage{enumitem}
\usepackage{xfrac}
\usepackage{slashed}
\usepackage{cancel}

\usepackage{afterpage}

\newcommand\blankpage{%
    \null
    \thispagestyle{empty}%
    \addtocounter{page}{-1}%
    \newpage}

%-------------------------------------------------------------------------------

\allowdisplaybreaks[4]

%-------------------------------------------------------------------------------

\newcommand{\mean}[1]{\ensuremath{\langle #1 \rangle}}
\newcommand{\kett}[1]{\ensuremath{\left | #1 \right \rangle}}
\newcommand{\brat}[1]{\ensuremath{\left \langle  #1 \right | }}
\newcommand{\ket}[1]{\left | #1 \right \rangle}
\newcommand{\bra}[1]{\left \langle  #1 \right |}

\newcommand{\nl}{\vskip 0.3cm}
\newcommand{\np}{\vskip 1.3cm}

\newcommand{\csi}{\xi}
\newcommand{\ro}{\rho}

\newcommand{\obar}[1]{\overline{#1}}
\newcommand{\ubar}[1]{\underline{#1}}
\newcommand{\psibar}{\bar{\psi}}
\newcommand{\sigmatilde}{\tilde{\sigma}}

\newcommand{\der}[2]{\frac{\partial #1}{\partial #2}}
\newcommand{\demu}[1]{\partial#1{\mu}}
\newcommand{\denu}[1]{\partial#1{\nu}}
\newcommand{\pder}[3]{\dfrac{\partial^{#1} #2}{\partial^{#1} #3}}
\newcommand{\de}[1]{\frac{d^2#1}{(2\pi)^2}}

\newcommand{\HRule}{\center{\rule{0.9\linewidth}{0.5mm}}}

\newcommand{\ssection}[2]{\vspace{0.5 cm} \section{ \texorpdfstring{\textbf{#1}}{#2} }}
\newcommand{\ssubsection}[1]{\vspace{0.2cm} \subsection{#1}}

\DeclareMathOperator{\sign}{sign}
\DeclareMathOperator{\Tr}{Tr}
\DeclareMathOperator{\tr}{tr}



%opening
\title{Modello di Gross-Neveu}
\author{Federico Tonielli}

\begin{document}

\begin{titlepage}
% \vspace{2 cm}
  \begin{center}
  \textsc{ \LARGE \textbf{QPT: generalità e modello di Ising in campo trasverso}  } \\[1 cm]
  \Large{Seminario d'esame di Transizioni di Fase - \\
  A.A. 2013-2014}
  \vspace{0.5 cm}
  \\[1.5 cm]
  \begin{flushright} 
  \Large 
  Candidato:\\
  \textsc{Federico Tonielli}
  \end{flushright}
  \vspace{2.5 cm}
  \includegraphics[width=12cm]{cherubino.pdf}
  \vfill
  {\Large 11 Settembre 2014}
  \end{center}  
\end{titlepage}

\blankpage

\tableofcontents
\blankpage

\ssection{Introduzione}{Introduzione}
L'argomento del seminario è la fenomenologia delle QPT confrontata con i risultati esatti ottenibili dal modello di Ising 1D in campo trasverso. Ci si concentra su \textbf{Q-C mapping} e su \textbf{crossover a T>0}, che è osservabile sperimentalmente; sarà usato il modello di Ising per capire il significato del primo e del secondo.

\ssection{Generalità}{Generalità}

\ssubsection{cosa sono le QPT e i QCP}
\begin{itemize}
 \item H dipende da un parametro adimensionale g; ad esempio \[ H(g) = H_0 + g H_1\]
 \item La funzione di partizione canonica è \[ Z(\beta,g) = Tr\left[ e^{-\beta H(g)}\right] \] L'energia libera è al solito 
 	\[ f(\beta,g) = -\frac{1}{\beta}\log Z \]
 \item Le CPT sono sempre governate da \emph{fluttuazioni termiche}: si verificano al variare di $\beta$ dove le funzioni termodinamiche sono non analitiche; 
 \item Le QPT sono governate da \emph{fluttuazioni quantistiche}: si verificano a $T=0$ (no fluttuaz. termiche) al variare di $g$ dove $E_{GS}(g)$ è non analitica; sono dovute a competizione di operatori non commutanti; \\ (le PT di sistemi quantistici a $T_c \neq 0$ sono invece governate solo da fluttuazioni termiche);
 \item ci interessano le QPT del II ordine; si verificano ad es. se $H_0$ e $H_1$ hanno GS con simmetrie diverse;
 \item il QCP è il punto nello spazio dei parametri in cui c'è la QPT del II ordine.
\end{itemize}

\ssubsection{differenza 1: QCP e dinamica}

\begin{itemize}
 \item In analogia con la teoria delle PT del II ordine per sistemi classici, ci si aspetta la divergenza delle lunghezze di correlazione e il conseguente scaling intorno al QCP: 
	\[ \csi \sim \Lambda^{-1} |g-g_c|^{-\nu}\]
	\[\csi_t \sim J^{-1} |g-g_c|^{-z\nu} \Rightarrow \csi_t \sim \csi^z\] dve J è una scala microscopica di energia e z è l'esponente critico dinamico;
 \item nel caso classico la parte cinetica e quella potenziale si fattorizzano in Z, quindi il punto critico d'equilibrio, che è determinato dalla seconda, non ha alcuna informazione sulla prima: per trovare z serve una teoria di non equilibrio;
 \item nel caso quantistico invece il QCP di equilibrio determina anche z, perché:
 		\begin{itemize}
 		 \item parte cinetica e potenziale non commutano, quindi la traccia non ne fattorizza i contributi; $e^{-\beta H}$ è un propagatore in un tempo immaginario $it \to \beta$, quindi la traccia si può scrivere come path integral su tutte traiettorie in tempo immaginario da $\tau=0$ a $\tau=\beta$ periodiche (perché $tr[U(\beta)] = \sum \bra{\psi}U(\beta)\ket{\psi}$);
 		 \item lo stato termico di equilibrio permette quindi di calcolare anche dei correlatori lungo il tempo immaginario, legati a quelli nel tempo reale da un prolungamento analitico (o a quelli nelle frequenze reali dalla trasformata di laplace): 
 		 \[\phi(\tau) = e^{H\tau}\phi e^{-H\tau} \Rightarrow G(\tau) = \mean{\phi^+(\tau)\phi(0)} = \sum_{n} e^{-(E_n-E_0)\tau} \left|\bra{n}\phi\ket{0}\right|^2\] 
 		\end{itemize}
 \item da questo si capiscono tre cose:
		\begin{itemize}
		 \item gli effetti quantistici (non commutatività di operatori) consistono in correlazioni lungo la direzione in più; a T=0 è infinitamente estesa e $\csi_{\tau}$ lungo di essa può divergere, producendo la divergenza di $\csi_t$; il QCP può quindi determinare z e si vede dalla formula che \[\csi_t \sim \Delta_{GAP}^{-1}\] (aggiungere $\sim \Delta^{-1}$ alla formula già scritta per $\csi_t$);
		 \item intorno a CP con $T\neq 0$ gli effetti quantistici non contano perché $\csi_t$ diverge e $\csi_{\tau}$ non può essendo limitato da $\beta_c$;
		 \item si capisce che a $T>0$ trovare la dinamica col path integral è non banale perché $G(\tau)$ è definito solo per $0\leq \tau \leq \beta$ e la trasformata di laplace non è calcolabile; \\per effettuare il prolungamento analitico servirebbe la forma funzionale esatta di $G(\tau)$ ma questo rende inservibile la teoria perturbativa (vero anche a $T=0$); \\in ogni caso per $T>0$ servono necessariamente teorie ad hoc (come nel caso classico).
		\end{itemize}
\end{itemize}

\ssubsection{QFT equivalente e Q-C mapping intorno al QCP}
\begin{itemize}
 \item Per quanto visto, le proprietà di scaling intorno al QCP si studiano a piccoli $\Delta$ e grandi $\csi$, cioè $\frac{\Delta}{J}\to 0$, $\csi\Lambda\to\infty$ e rapporti di x, $\omega$, T, ecc. con $\Delta$ e $\csi$ fissati; \\questo è come mandare $J,\Lambda \to \infty$ e contemporaneamente $g\to g_c$ per mantenere i rapporti finiti; \\questi limiti definiscono una QFT rinormalizzata nel continuo, niente altro che una teoria euclidea (=SFT classica) in dimensione $D=d+1$;
 \item il legame tra il limite universale piccoli $\Delta$ e grandi $\csi$ e il limite del continuo sarà mostrato nel modello di Ising come caso esplicito
 \item Questo Q-C mapping si costruisce a partire dal path integral sul tempo immaginario, quindi ne ha tutte le limitazioni a $T>0$;
 \item Una volta costruito o assunto in modo fenomenologico si possono usare a $T=0$ tutti gli strumenti classici (naive power counting, MF, RG benché anisotropo e tecnicamente difficile da calcolare, $\varepsilon$-expansion).
\end{itemize}

\ssubsection{differenza 2: scaling al QCP}
\begin{itemize}
 \item Dal momento che la direzione temporale cambia di $b^z$ sotto rescaling di quella spaziale di $b$, f (che è $-\frac{1}{\beta V}\dots$) ha dimensioni effettive $d+z$; la sua forma di scala è \[f(\lambda,K) \sim b^{-d-z} f(b^{1/\nu}\lambda,b^{y_K}K)\,\,\,\,\,\,\,\,\,\,\,\,\,\,\,\,(\lambda = \frac{g-g_c}{g_c}\text{, K irrilevanti})\] intorno al QCP quindi la SFT effettiva ha dimensione d+1 ma i fenomeni critici sono quelli di un sistema classico in dimensione d+z perché, a differenza del caso classico, c'è un'anisotropia intrinseca.
\end{itemize}

\ssubsection{crossover a T>0}
\begin{itemize}
 \item la condizione $\csi_t \gg/\ll \beta$ o l'equivalente (ma con significato fisico più evidente) $\|Delta| \ll/\gg kT$ determinano rispettivamente la zona in cui contano solo le fluttuazioni termiche (classica) e quella in cui contano anche le quantistiche (quantum critical; eventuali CPT sono nella zona classica;\\ \emph{disegno di un crossover a T>0 con e senza CPT a T>0};
 \item nelle zone classiche
 \item con $T>0$ siamo nel caso finite size; il reciproco della lunghezza totale nella direzione finita (che nel nostro caso è $\beta^{-1}=T$) sotto rescaling si comporta come una variabile di scala con dimensione di scala +1, quindi posso estendere la forma di scala a $T\neq 0$ (finite-size scaling): 
 	\[ f(\lambda,T,K) \sim b^{-d-z} f(b^{1/nu}\lambda,bT,b^{y_K}K) \]
 	\begin{align*} &G(|x|\to\infty,\lambda,T) \sim b^{2d_{\phi}} G(b^{-1}|x|,b^{1/nu}\lambda,bT) \Rightarrow \\ &\Rightarrow G(|x|\to\infty,\lambda,T)=CT^{-2d_{\phi}}\Phi(T|x|, T^{-z}\Delta)\end{align*}
\end{itemize}

\ssection{Modello di Ising 1D in campo trasverso}{Modello di Ising 1D in campo trasverso}
\ssubsection{Il modello}
\begin{itemize}
 \item L'hamiltoniana parametrica è (con PBC) \[H(g) = -J\sum_{i=1}^N\left[\sigma_i^z\sigma_{i+1}^z + g\sigma_i^x\right]\] 
 \item I GS dei due termini hanno proprietà di simmetria diverse sotto $\ket{\uparrow} \to \ket{\downarrow}$ (coniugio per $\sigma_{tot}^x$); \\per $g\ll 1$ il GS è ~ $\ket{\uparrow\uparrow\cdots}$, la fase è ferromagnetica: ha $\mean{\sigma^z}\neq 0$ e $\mean{\sigma^x}=0$; per $g\gg 1$ il GS è $\ket{\rightarrow\rightarrow\cdots}$, la fase è paramagnetica quantistica: l'accoppiamento paramagnetico vince su quello a primi vicini e le medie sono al contrario rispetto al caso ferromagnetico.
\end{itemize}

\ssubsection{Spettro esatto (cenni)}
\begin{itemize}
 \item Jordan-Wigner e hamiltoniana
 \item (occhio al numero di fermioni, spiegare cosa non va)
 \item Trasformata di Fourier e hamiltoniana
 \item Trasformata di Bogolubov e hamiltoniana
 \item Il GS è lo stato con 0 fermioni $\gamma_k$, vedere che il gap si chiude, (vedere che l'energia del GS è non analitica in g ???), vedere che $z\nu=1$;
 \item Le eccitazioni rispetto al GS sono quelle create dagli operatori $\gamma_k^+$, ma non hanno un'ovvia interpretazione in termini di DW tra regioni con magnetizzazione opposta o di spin flip; solo nei due limiti $g\to 0$ e $g \to \infty$ si recupera la descrizione euristica in termini di DW e spin flip, (vederlo???) 
\end{itemize}

\ssubsection{Limite del continuo e QFT corrispondente}
\begin{itemize}
 \item Hamiltoniana in termini di $\Psi$; il limite del continuo per la teoria di campo e quello universale coincidono;
 \item le relazioni di scaling producono $z=1$ e $\nu=1$;
 \item il campo $\sigma^z$ può acquistare dimensione anomala anche se la teoria è di fermioni liberi perché è un prodotto di infiniti operatori di campo fermionici;
  \item le correzioni con a finito (gradienti di ordine superiore, cosa altro?) ed eventuali altre (ad es. $\sigma^x_i \sigma^x_{i+1}$) sono irrilevanti al QCP per power counting.
\end{itemize}

\ssubsection{Crossover a T>0 per il modello di Ising}
\begin{itemize}
 \item per calcolare il correlatore $\sigma^z_i\sigma^z_{i+n}$ si usa questo: 
      \begin{itemize} 
      \item è la media di un prodotto di operatori di Majorana, dei quali so la funzione di Green termica perché si calcola facilmente con lo spettro esatto, fermi-dirac e la trasformazione diagonalizzante (non nel limite del continuo!);
      \item vale il teorema di Wick per la media termica di prodotti operatori fermionici liberi;
      \item lemmi tecnici (per scrivere la somma di contrazioni come un determinante di Toepliz, il limite asintotico $n\to\infty$ di questo, il limite sotto forma di integrale di una funzione);
      \end{itemize}
 \item il risultato in forma esplicita è [...]; in tutti i casi, il correlatore tende a 0 per $|x|\to\infty$: a $T>0$ non c'è ordine a lungo raggio di $\sigma^z$, come  ci si aspetta (è nella classe di universalità del modello di ising 1D con interazioni a corto raggio);
 \item il finite-size scaling è rispettato e la dimensione anomala di $\sigma^z$ è $\eta = $;
 \item si discute il crossover:
      \begin{itemize}
      \item nella fase paramagnetica anche a $T=0$ non c'è LRO e quindi giustamente $\csi$ tende a una costante: \[\Delta/T \to +\infty \Rightarrow \csi \sim \text{cost.} + e^{-\Delta/T}\] 
      \item nella fase ferromagnetica c'è LRO a $T=0$ e $\csi$ diverge: \[\Delta/T \to -\infty \Rightarrow \csi \sim e^{-|\Delta|/T}\]
      \item nella fase quantum-critical  $\csi$ va con una legge a potenza (in questa fase è come se fosse $\Delta=0$): \[\Delta=0 \Rightarrow \csi \sim 1/T \]
      \end{itemize}
 \item le due fasi chiamate ``classiche'' lo sono perché: 
      \begin{itemize}
       \item le eccitazioni rispetto al GS sono quasiparticelle fermioniche $\gamma_k$ con energia $\varepsilon_k \simeq \sqrt{\Delta^2 + c^2k^2} \simeq |\Delta| + \frac{c^2K^2}{2|\Delta|}$ per k piccolo ($\Delta$ è fissato), quindi sono libere con massa $m\sim \Delta$;
       \item se si assume in modo euristico che $\csi$ sia anche dell'ordine della distanza media tra queste quasiparticelle eccitate termicamente, allora la lunghezza d'onda termica $\lambda_T \sim \frac{1}{(|\Delta|T)^{1/2}}$ è molto più piccola di $\csi$, separazione media; 
       \item ($\lambda_T$ è calcolata assumendo vero il teorema di equipartizione classico, quindi è un conto autoconsistente; una teoria classica per la dinamica, ottenuta elaborando questa osservazione da Sachdev [vedi libro], è in perfetto accordo con le simulazioni numeriche).
      \end{itemize}

\end{itemize}


\end{document}
