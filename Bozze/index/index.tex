\documentclass[a4paper,10pt]{article}

\usepackage{amsmath, amssymb, amsthm}
\usepackage{mathtools}
\usepackage{vmargin}

\usepackage[all]{xy}
\usepackage{graphicx}
\usepackage{framed}

\usepackage[utf8]{inputenc}
\usepackage[T1]{fontenc}   
\usepackage[english]{babel}
% \usepackage{fullpage}
\usepackage{hyperref}

\usepackage{enumitem}
\usepackage{xfrac}
\usepackage{slashed}
\usepackage{color}
\usepackage{cancel}

\usepackage{afterpage}

\newcommand\blankpage{%
    \null
    \thispagestyle{empty}%
    \addtocounter{page}{-1}%
    \newpage}

%-------------------------------------------------------------------------------

\allowdisplaybreaks[4]

%-------------------------------------------------------------------------------

\newcommand{\ssection}[2]{\section{ \texorpdfstring{\textbf{#1}}{#2} }}

\newcommand{\HRule}{\center{\rule{0.9\linewidth}{0.5mm}}}

% add these definistions to the definitions' file!!!!!!! --------------------------------------

\DeclareMathOperator{\sign}{sign}
\DeclareMathOperator{\Tr}{Tr}
\DeclareMathOperator{\tr}{tr}
\DeclareMathOperator{\Det}{Det}

\theoremstyle{remark}
\newtheorem{remark}{Remark}

\newcommand{\mean}[1]{\ensuremath{\langle #1 \rangle}}
\newcommand{\kett}[1]{\ensuremath{\left | #1 \right \rangle}}
\newcommand{\brat}[1]{\ensuremath{\left \langle  #1 \right | }}
\newcommand{\ket}[1]{\left | #1 \right \rangle}
\newcommand{\bra}[1]{\left \langle  #1 \right |}
\newcommand{\ro}{\rho}
\newcommand{\ra}{\rightarrow}

\newcommand{\nline}{\\[0.3cm]}
\newcommand{\nlspace}{\vskip 0.3cm}
\newcommand{\nsec}{\vskip 0.8cm}
\newcommand{\npar}{\vskip 1.3cm}

\newcommand{\obar}[1]{\overline{#1}}
\newcommand{\ubar}[1]{\underline{#1}}
\newcommand{\psibar}{\bar{\psi}}
\newcommand{\sigmatilde}{\tilde{\sigma}}

\newcommand{\der}[2]{\frac{\partial #1}{\partial #2}}
\newcommand{\demu}[1]{\partial#1{\mu}}
\newcommand{\denu}[1]{\partial#1{\nu}}
\newcommand{\pder}[3]{\dfrac{\partial^{#1} #2}{\partial^{#1} #3}}
\newcommand{\de}[1]{\frac{d^2#1}{(2\pi)^2}}

\newcommand*{\mathcolor}{}  %allows to change the colour inside math mode without destroying the correct spacing like with \textcolor
\def\mathcolor#1#{\mathcoloraux{#1}}
\newcommand*{\mathcoloraux}[3]{%
  \protect\leavevmode
  \begingroup
    \color#1{#2}#3%
  \endgroup
}
 
% ---------------------------------------------------------------------------------------------


% ---- info ---------
\title{Outline of possible contents}
\author{Federico Tonielli}
% ---- document -----

\begin{document} 

 \section{Intro}
  \begin{enumerate}
   \item the whole theory will be developed for fermions; a literature for bosons is already available and here some small technical adjustments are carried out following the same programme
   \item the first question can well be: what does ``non equilibrium'' mean? The answer will be given later in the 3rd chapter
   \item another can be: why non-equilibrium? why non-equilibrium fermions? Among all intriguing phenomena which are exclusive non-eq. features (examples? No), the one on which we will focus is the possibility to engineer states of matter with interesting pairing and topological properties 
   \item in this thesis we shall explain what is the mechanism behind this, how to build effective quadratic theories (BdG-like) to describe these states, what is the field-theoretic analog of this Mean Field mapping and we will explicitly evaluate corrections to MF
  \end{enumerate}
 \nsec
 
 \section{Dynamics of Open Quantum systems}
  \subsection{General and Markovian Open Quantum systems}
   \begin{enumerate}
    \item Description of open quantum systems using a density matrix and not a pure state; problems of finding a satisfactory dynamical equation for it
    \item QME for Markovian systems; discussion of the physical setting in which it approximately holds (i.e. of the approximations to be made to derive it and of the systems for which they hold)
   \end{enumerate}
  \subsection{QME, derivation}
   Derivation of the QME tracing out bath degrees of freedom in 2nd order time-dependent PT (maybe in an appendix?) 
 \nsec  
 
 \section{From QME to Keldysh Field Theory}
  \subsection{Quadratic QME to Quadratic NEqFT}
   \begin{enumerate}
    \item Derivation of the Keldysh action in the $\pm$ basis  via coherent states decomposition and rotation to the bosonic-like RAK one
    \item Meaning of Green's functions expressing them as operator averages, prelude to ordering issues (as it is expressed in my derivation of Eisert and Prosen's results)
    \item What is Non Equilibrium? = violation of FDT in our language (?, Yes); example for it (?, No)
   \end{enumerate}
  \subsection{Interacting QME to Interacting NEqFT}
   \begin{enumerate}
    \item Ordering issue in T=0 QFT and point-splitting regularization as a way to solve it 
    \item Derivation of the properly regularized Keldysh action tracing out bath degrees of freedom and then taking the Markov-RWA limit
    \item Alternative derivation via coherent states decomposition      
   \end{enumerate}
  \subsection{Effective Action}
   \begin{enumerate}
    \item Definition, scope and path-integral expression
    \item Exact propagator with a Keldysh structure always corresponds to a quadratic H and set of linear $L_i$ (?, it can be used afterwards to suggest that an approximation to $\Gamma^{(2)}$ can always be used to approximate the Lindblad dynamics) 
    \item Discussion of symmetries in the NEq context (?, it's crucial that our system lacks of time-reversal invariance, but then we should also mention continuous symmetries because it has a global U(1) and then also Nöther's theorem.. No)
   \end{enumerate}
  \nsec 
  
  \section{Driven-Dissipative fermions}
   \subsection{Why?}
    \begin{enumerate}
     \item Idea behind engineered dissipation
     \item Model under study: $p_x+ip_y$ superconductor ground state (not the only interesting state realizable!)
    \end{enumerate}
   \subsection{Physical Setting}
    \begin{enumerate}
     \item How to implement the set of quadratic Lindblads
     \item NESS for this set and its equivalence with the state of interest
    \end{enumerate}
   \subsection{Dissipative MFT}
    DMFT as an approximate mapping from quartic (late) dynamics to quadratic dynamics
  \nsec 
  
  \section{Field-theoretic treatment of the problem} 
   \subsection{Self-Consistent 1-loop improved Schwinger-Dyson equation}
    \begin{enumerate}
     \item Schwinger-Dyson (formulated via the effective action formalism) identified as the FT analog of DMFT
     \item G actually solves the equation (discussion on the way of solving it still missing: does it force the structure of G in any case? is the solution to it unique actually?)
     \item The set of linear $L_i$ corresponding to G is actually the one already found (not verified, but strongly believe so: one can obtain this set given G going backwards through the way he could obtain G given the set)
    \end{enumerate} 
   \subsection{Corrections to DMFT}
    \begin{enumerate}
     \item Evaluation of all two-loops diagrams correcting G
     \item What is the small parameter in the loop expansion?
     \item Can we identify where factorization and purity of the NESS enter in the two-loop diagrams?
    \end{enumerate}

  \section{Great absent}
    What changes if we add Hamiltonian/Lindblad perturbations? (of course it is absent: it would require additional work! However, the first question I will receive will be ``physical systems have also Hamiltonian dynamics...'' I \underline{must} have something to tell about it.)
    
\end{document}
